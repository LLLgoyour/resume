% !TEX TS-program = xelatex
% !TEX encoding = UTF-8 Unicode
% !Mode:: "TeX:UTF-8"

\documentclass{resume}
\usepackage{zh_CN-Adobefonts_external}
\usepackage{linespacing_fix}
\usepackage{cite}

\begin{document}
\pagenumbering{gobble}

\name{Jack D.}
\contactInfo{(+1) 207-776-4386}{lllgoyour@outlook.com}{个人开发者}{GitHub @lllgoyour}{Blog https://lllgoyour.tk}

\section{个人总结}
我喜欢尝试不同、新奇的技术和编程语言。作为学生和个人开发者参加过和伙伴、团队合作的项目,包括运维、编写客户端、编写前端、重构后端等事项。我对Web技术、软件开发有浓厚兴趣。\textbf{现在只是名学生兼个人开发者。}

% \section{\faGraduationCap\ 教育背景}
\section{教育背景}
\datedsubsection{\textbf{Kimball Union Academy},\textit{在读高中生}}{2020.9 - 至今}
\ \textbf{排名11/133(前10\%)},预计2015年6月毕业
\datedsubsection{\textbf{北京理工大学},软件工程,\textit{工学学士}}{2011.9 - 2015.6}
\ \textbf{排名2/62(前5\%)},国家励志奖学金,人民奖学金(7次),科技竞赛奖(2次),北京市普通高等学校优秀毕业生,北京理工大学优秀毕业生,软件学院金牌毕业生,优秀团员/优秀学生(5次)
\datedsubsection{\textbf{荷兰 莱顿大学},计算机科学与技术,\textit{国家留学基金委公派交换生}}{2015.3 - 2015.5}
\ 2014年中国政府奖学金(\textit{http://www.csc.edu.cn/}),DID-ACTE项目交换生(\textit{http://did-acte.org/})

% \section{\faCogs\ IT 技能}
\section{技术能力}
% increase linespacing [parsep=0.5ex]
\begin{itemize}[parsep=0.2ex]
  \item \textbf{编程语言}: JavaScript (ECMAScript, Node.js), HTML/CSS, Python, Go, SQL, C, Shell
  \item \textbf{操作系统,数据库与工程构建}: Linux/macOS/MySQL/MongoDB/Git/webpack/Progressive Web App
  \item \textbf{关键词}: React/Vue.js/D3.js(SVG)/three.js(canvas, WebGL)/chrome extension/Express
\end{itemize}

% \end{itemize}

\section{实习经历}
\datedsubsection{\textbf{阿里巴巴集团 | Alibaba}, 前端开发工程师}{2017.6-2017.9}
\begin{itemize}
%   \item 飞猪北京前端团队全面负责各交通线的票务(机票/火车票/汽车票) web 应用与事业群基础架构研发
  \item 独立负责车站地图开发(React),通过HTML5 本地存储及JSBridge实现在阿里全系应用中发布上线
  \item 独立负责BU SPM chrome插件开发,支付成功/订单详情等页面的开发与交叉营销的接入工作
\end{itemize}


% \begin{onehalfspacing}
% \end{onehalfspacing}

% \datedsubsection{\textbf{DID-ACTE} 荷兰莱顿}{2015年3月 - 2015年6月}
% \role{本科毕业设计}{LIACS 交换生}
% 利用结巴分词对中国古文进行分词与词性标注,用已有领域知识训练形成 classifier 并对结果进行调优
% \begin{onehalfspacing}
% \begin{itemize}
%   \item 利用结巴分词对中国古文进行分词与词性标注
%   \item 利用已有领域知识训练形成 classifier, 并用分词结果进行测试反馈
%   \item 尝试不同规则,对 classifier 进行调优
% \end{itemize}
% \end{onehalfspacing}

\section{竞赛获奖/项目作品}
% increase linespacing [parsep=0.5ex]
\begin{itemize}[parsep=0.2ex]
%   \item LeetCodeOJ Solutions, \textit{https://github.com/hijiangtao/LeetCodeOJ}
  \item 第三届中国软件杯大学生软件设计大赛\textbf{全国一等奖}( \textit{http://www.cnsoftbei.com/} ),2014 年8月
  \item 中国机器人大赛创意设计大赛\textbf{全国特等奖}( \textit{http://www.rcccaa.org/} ),2013年8月
%   \item 中国机器人大赛暨Robocup公开赛(武术擂台赛)全国一等奖,2013年10月
  \item 第11届北京理工大学“世纪杯”竞赛学生课外科技作品竞赛\textbf{特等奖},2013年8月
  \item VIS Components for security system, \textit{https://hijiangtao.github.io/ss-vis-component/}
  \item 个人博客:\textit{https://hijiangtao.github.io/},更多作品见 \textit{https://github.com/hijiangtao}
%   \item 电视节目"爸爸去哪儿"可视化分析展示, \textit{https://hijiangtao.github.io/variety-show-hot-spot-vis/}
\end{itemize}

% \section{\faHeartO\ 项目/作品摘要}
% \section{项目/作品摘要}
% \datedline{\textit{An Integrated Version of Security Monitor Vis System}, https://hijiangtao.github.io/ss-vis-component/ }{}
% \datedline{\textit{Dark-Tech}, https://github.com/hijiangtao/dark-tech/ }{}
% \datedline{\textit{融合社交网络数据挖掘的电视节目可视化分析系统}, https://hijiangtao.github.io/variety-show-hot-spot-vis/}{}
% \datedline{\textit{LeetCodeOJ Solutions}, https://github.com/hijiangtao/LeetCodeOJ}{}
% \datedline{\textit{Info-Vis}, https://github.com/ISCAS-VIS/infovis-ucas}{}


% \section{\faInfo\ 社会实践/其他}
\section{社区参与/实践其他}
% increase linespacing [parsep=0.5ex]
\begin{itemize}[parsep=0.2ex]
  \item 乐于参与开源社区讨论,\textbf{参与翻译 Vue.js, webpack, WebAssembly, Babel 文档,印记中文成员}
  \item 中国科学院大学2016秋季学期可视化与可视分析课程助教,\textit{http://vis.ios.ac.cn/infovis-ucas/}
  \item 未来论坛学生会成员、北理社联新闻信息中心主任、北理工软件学院学生会宣传部副部长(2012-2016)
  \item 2013-2015 北京市共青团“温暖衣冬”志愿者,第九届园博会志愿者,2014 FLL机器人世锦赛志愿者
\end{itemize}

%% Reference
%\newpage
%\bibliographystyle{IEEETran}
%\bibliography{mycite}
\end{document}
